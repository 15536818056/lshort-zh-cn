%%%%%%%%%%%%%%%%%%%%%%%%%%%%%%%%%%%%%%%%%%%%%%%%%%%%%%%%%%%%%%%%%
% Contents: Who contributed to this Document
% $Id: overview.tex,v 1.2 2003/03/19 20:57:46 oetiker Exp $
%%%%%%%%%%%%%%%%%%%%%%%%%%%%%%%%%%%%%%%%%%%%%%%%%%%%%%%%%%%%%%%%%
% 中文~4.20~翻译:zpxing@bbs.ctex  email: zpxing at gmail dot com
%%%%%%%%%%%%%%%%%%%%%%%%%%%%%%%%%%%%%%%%%%%%%%%%%%%%%%%%%%%%%%%%%

% Because this introduction is the reader's first impression, I have
% edited very heavily to try to clarify and economize the language.
% I hope you do not mind! I always try to ask "is this word needed?"
% in my own writing but I don't want to impose my style on you...
% but here I think it may be more important than the rest of the book.
% --baron

%\chapter{Preface}
\chapter{前言}

%\LaTeX{} \cite{manual} is a typesetting system that is very suitable
%for producing scientific and mathematical documents of high
%typographical quality. It is also suitable for producing all sorts
%of other documents, from simple letters to complete books. \LaTeX{}
%uses \TeX{} \cite{texbook} as its formatting engine.
\LaTeX{}~\cite{manual}~是一种排版系统,它非常适用于生成高印刷质量
的科技和数学类文档。这个系统同样适用于生成从简单信件到完整书籍
的所有其他种类的文档。\LaTeX~使用~\TeX{}\cite{texbook}~作为它的
格式化引擎。%

%This short introduction describes \LaTeXe{} and should be sufficient
%for most applications of \LaTeX. Refer to~\cite{manual,companion}
%for a complete description of the \LaTeX{} system.
这份短小的介绍描述了~\LaTeXe{}~的使用,对~\LaTeX{}~的大多数应用来说
应该是足够了。参考文献~\cite{manual,companion}~对~\LaTeX{}~系统
提供了完整的描述。%

\bigskip
%\noindent This introduction is split into 6 chapters:
\noindent 这份介绍共有六章:%
%\begin{description}
%\item[Chapter 1] tells you about the basic structure of \LaTeXe{}
%  documents. You will also learn a bit about the history of \LaTeX{}.
%  After reading this chapter, you should have a rough understanding how
%  \LaTeX{} works.
%\item[Chapter 2] goes into the details of typesetting your
%  documents. It explains most of the essential \LaTeX{} commands and
%  environments. After reading this chapter, you will be able to write
%  your first documents.
%\item[Chapter 3] explains how to typeset formulae with \LaTeX. Many
%  examples demonstrate how to use one of \LaTeX{}'s
%  main strengths. At the end of the chapter are tables listing
%  all mathematical symbols available in \LaTeX{}.
%\item[Chapter 4] explains indexes,  bibliography generation and
%  inclusion of EPS graphics. It introduces creation of PDF documents with pdf\LaTeX{}
%  and presents some handy extension packages.
%\item[Chapter 5] shows how to use \LaTeX{} for creating graphics. Instead
%  of drawing a picture with some graphics program, saving it to a file and
%  then including it into \LaTeX{} you describe the picture and have \LaTeX{}
%  draw it for you.
%\item[Chapter 6] contains some potentially dangerous information about
%  how to alter the
%  standard document layout produced by \LaTeX{}. It will tell you how  to
%  change things such that the beautiful output of \LaTeX{}
%  turns ugly or stunning, depending on your abilities.
%\end{description}
\begin{description}%
\item[第一章] 告诉你关于~\LaTeXe{}~文档的基本结构。你也会从中了解一点~\LaTeX{}~的历史。
              阅读这一章后,你应该对~\LaTeX{}~如何工作有一个大致的理解。
\item[第二章] 探究文档排版的细节。它解释了大部分必要的~\LaTeX{}~命令和环境。在阅
              读完这一章之后,你就能够编写你的第一份文档了。%
\item[第三章] 解释了如何使用~\LaTeX{}~排版公式。同时,大量的例子会有助于你理解
              ~\LaTeX{}~是如何的强大。在这个章节的结尾,你会找到列出~\LaTeX{}~
              中所有可用数学符号的表格。%
\item[第四章] 解释了索引和参考文件的生成、EPS~图形的插入。它介绍了如何使用~pdf\LaTeX{}~生成~pdf~文档和一些其他有用的扩展宏包。%
\item[第五章] 演示如何使用~\LaTeX{}~创建图形。不必使用图形软件画图、存盘并插入~\LaTeX{}~文档,你可以直接描述
              图形,然后~\LaTeX{}~会替你画好它。
\item[第六章]
包含一些潜在的危险信息,内容是关于如何改变~\LaTeX{}~所产生文档的标准布局。它会告诉你如何把~\LaTeX{}~的输出
变得更糟糕,或者更上一层楼,当然这取决于你的能力。
\end{description}%

\bigskip
%\noindent It is important to read the chapters in order---the book is
%not that big, after all. Be sure to carefully read the examples,
%because a lot of the information is in the
%examples placed throughout the book.
\noindent 按照顺序阅读这些章节是很重要的
\pozhehao 这本书毕竟不长。一定要认真阅读例子,因为在贯穿全篇的各种例子里包含了很多的信息。%

\bigskip
%\noindent \LaTeX{} is available for most computers, from the PC and Mac to large
%UNIX and VMS systems. On many university computer clusters you will
%find that a \LaTeX{} installation is available, ready to use.
%Information on how to access
%the local \LaTeX{} installation should be provided in the \guide. If
%you have problems getting started, ask the person who gave you this
%booklet. The scope of this document is \emph{not} to tell you how to
%install and set up a \LaTeX{} system, but to teach you how to write
%your documents so that they can be processed by~\LaTeX{}.
\noindent
\LaTeX{}~适用于从~PC~和~Mac~到大型的~UNIX~和~VMS~系统上。许多大学的计算机集群上安
装了~\LaTeX{},随时可以使用。\guide~里应该会介绍如何使用本地安装的~\LaTeX{}。
如果有问题,就去问给你这本小册子的人。这份文档\emph{不}会告诉你如何安装一个~\LaTeX{}~系统,
而是教会你编写~\LaTeX{}~能够处理的文档。

\bigskip
%\noindent If you need to get hold of any \LaTeX{} related material,
%have a look at one of the Comprehensive \TeX{} Archive Network
%(\texttt{CTAN}) sites. The homepage is at
%\texttt{http://www.ctan.org}. All packages can also be retrieved from
%the ftp archive \texttt{ftp://www.ctan.org} and its mirror
%sites all over the world.
\noindent 如果你想取得~\LaTeX{}~的相关材料,请访问“Comprehensive
\TeX{} Archive Network”
~(\texttt{CTAN})~站点,主页是~\texttt{http://www.ctan.org}。所有的宏包
也可以从~ftp~归档站点~\texttt{ftp://www.ctan.org}~和遍布全球的各个镜像站点中获得。
所有的宏包都可以在~\texttt{ftp://ctan.tug.org}~以及它遍布全球的镜像取得。

%You will find other references to CTAN throughout the book, especially
%pointers to software and documents you might want to download. Instead
%of writing down complete urls, I just wrote \texttt{CTAN:} followed by
%whatever location within the CTAN tree you should go to.
在本书中你会找到其他引用~\texttt{CTAN}~的地方,尤其是,给出你可能需要下载的软件和文档的
指示。这里没有写出完整的~url,而仅仅是其在~\texttt{CTAN:}~之后的树状结构中的位置。%


%If you want to run \LaTeX{} on your own computer, take a look at what
%is available from \CTAN|systems|.
请先看看~\CTAN|systems|~中有些什么,如果你想在自己的计算机上运行~\LaTeX{}。%

\vspace{\stretch{1}}
%\noindent If you have ideas for something to be
%added, removed or altered in this document, please let me know. I am
%especially interested in feedback from \LaTeX{} novices about which
%bits of this intro are easy to understand and which could be explained
%better.
\noindent 如果你有意在这份文档中增加、删除或者改变一些内容,请通知我。我对~\LaTeX{}~
初学者的反馈特别感兴趣,尤其是关于这份介绍哪些部分很容易理解,哪些部分可能需要更好地解释。%
%

\bigskip
\begin{verse}
\contrib{Tobias Oetiker}{oetiker@ee.ethz.ch}%
\noindent{Department of Information Technology and\\ Electrical
Engineering,\\
Swiss Federal Institute of Technology}
\end{verse}
\vspace{\stretch{1}}

%\noindent The current version of this document is available on\\
%\CTAN|info/lshort|

\noindent 这份文档的最新版本在\\
\CTAN|info/lshort|\\%

\smallskip
\noindent 关于这份文档的最新中文翻译,请咨询\\
\href{http://bbs.ctex.org}{\texttt{http://bbs.ctex.org}}\\

\endinput



%

% Local Variables:
% TeX-master: "lshort2e"
% mode: latex
% mode: flyspell
% End:
