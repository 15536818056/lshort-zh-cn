\chapter{安装 \protect\TeX\ 发行版}\label{app:install}

\begin{intro}
高德纳的 \TeX\ 程序开发于 20 世纪 80 年代,那时候电子计算机的运算能力有限,\TeX\ 还是大型服务器上的玩物。
而如今个人计算机完全能够胜任排版的工作,并催生了用于个人计算机的工具集合—— \TeX\ 发行版的发展。

本章会简单介绍如何安装 \TeX\ 发行版,以及保持发行版的内容紧跟最新。后者非常重要,
因为 \LaTeX\ 宏包是不断更新换代的。
\end{intro}

\section{\protect\TeX\ 发行版简介}\label{sec:dists}

一个\textbf{\TeX\ 发行版}是 \TeX\ 排版引擎、支持排版的文件(基本格式、\LaTeX\ 宏包、字体等)以及一些辅助工具的集合。
各式各样的 \TeX\ 发行版经过十多年的发展,大浪淘沙,现今的两个主流发行版为:
\begin{itemize}
  \item \textbf{\TeX\ Live}\par
  \TeX\ Live 由类 UNIX 系统上的 te\TeX\ 发展并取而代之,最终成为跨平台的 \TeX\ 发行版。
  \TeX\ Live 自 2011 年起以年份作为发行版的版本号,保持了一年一更的频率。

  Mac\TeX\ 是 OS X 系统下的一个定制化的 \TeX\ Live 版本,与 \TeX\ Live 同步更新。

  \item \textbf{Mik\TeX}\par
  Mik\TeX\ 是主要用于 Windows 平台的一个稳定发展的 \TeX\ 发行版。
  中国的 \LaTeX\ 用户应该对“C\TeX\ 套装”比较熟悉,它是一个经过本地化配置的 Mik\TeX 。
\end{itemize}

\TeX\ Live 和 Mik\TeX\ 都集成了一个简单的 \LaTeX\ 源代码编辑器 \TeX works(Mac\TeX\ 则集成了类似的 \TeX shop)。
用户在完成发行版的安装后,可直接打开编辑器开始编写 \LaTeX\ 源代码。

\subsection{安装发行版}\label{subsec:install-dists}

\subsubsection{\TeX\ Live}

\TeX\ Live 在 \url{http://www.tug.org/texlive/} 上提供 ISO 光盘镜像%
\footnote{Linux 发行版的软件源也提供 \TeX\ Live 的安装,不过不够完整,更新也不是很及时。建议直接从镜像安装。}。
下载镜像到本地,挂载到虚拟光驱,或者用压缩工具解压后,在其根目录有几个用于安装的脚本:
\begin{itemize}
  \item 用于 Windows 的批处理文件:
  \begin{itemize}
    \item \texttt{install-tl-windows.bat} 双击启动图形界面安装程序(简单安装);
    \item \texttt{install-tl-advanced.bat} 双击启动图形界面安装程序(定制安装);
    \item 在命令提示符中输入 \texttt{install-tl-windows.bat -no-gui} 启动文本界面安装程序。
  \end{itemize}
  \item 用于 Linux 的 Perl 脚本 \texttt{install-tl} :
  \begin{itemize}
    \item \texttt{install-tl} 启动文本界面安装程序;
    \item \texttt{install-tl -gui=wizard} 启动图形界面安装程序(简单安装);
    \item \texttt{install-tl -gui=peritk} 启动图形界面安装程序(定制安装)。
  \end{itemize}
\end{itemize}

Linux 下 \TeX\ Live 安装完毕后,还需要在 root 权限下进行以下操作,使得 \texttt{xelatex} 命令能正确通过 \pkg{fontspec}
等宏包使用字体\footnote{\url{http://www.tug.org/texlive/doc/texlive-zh-cn/texlive-zh-cn.pdf},%
可用 \texttt{texdoc texlive-zh-cn} 在本地打开。}:
\begin{enumerate}
  \item 将 \texttt{texlive-fontconfig.conf} 文件复制到 \texttt{/etc/fonts/conf.d/09-texlive.conf}。
  \item 运行 \texttt{fc-cache -fsv}。
\end{enumerate}

\subsubsection{Mik\TeX}
从 Mik\TeX\ 官网 \url{http://www.miktex.org/} 下载名为 \texttt{basic-miktex-***.exe} 的 Windows 安装程序包。
下载后直接双击打开,按照程序的提示进行安装即可。

\section{安装和更新宏包}\label{sec:pkg-manager}

\TeX\ Live 提供了图形界面的宏包管理器 \TeX\ Live Manager 用于安装和更新宏包,而 Mik\-\TeX\ 也提供了管理器
Mik\-\TeX\ Package Manager。用户可直接打开程序,进行宏包的安装和更新
(Mik\TeX\ Package Manager 有普通权限和管理员权限的版本,建议总是打开管理员权限的程序)。

两者也可以通过各自的命令行工具安装更新宏包:
\begin{verbatim}
% TeX Live 命令行工具 tlmgr 的使用示例
tlmgr install <package-name> % 安装某个宏包
tlmgr remove <package-name>  % 卸载某个宏包
tlmgr update --all --self    % 更新所有宏包(包括 tlmgr 本身)
tlmgr update --list          % 列出所有可更新的宏包
tlmgr repository set http://.../CTAN/systems/texlive/tlnet
                             % 指定更新源(CTAN)地址
tlmgr info <package-name>    % 查看宏包信息
                             % 加 --list 选项可列出宏包的所有文件
\end{verbatim}

\begin{verbatim}
% MikTeX 命令行工具 mpm 的使用示例
% 建议始终加 --admin 选项使用
mpm --admin --install <package-name>   % 安装某个宏包
mpm --admin --uninstall <package-name> % 卸载某个宏包
mpm --admin --update                   % 更新所有宏包
mpm --admin --set-repository=http://.../CTAN/systems/win32/miktex/tm/packages
                                       % 指定更新源(CTAN)地址
mpm --admin --print-package-info <package-name>
                                       % 查看宏包信息
\end{verbatim}

\TeX\ Live 默认安装所有宏包,而 Mik\TeX\ 的安装程序只包含了若干用于 \LaTeX\ 的基本宏包。从 \TeX\ Live 的光盘镜像和 Mik\-\TeX\ 的安装包体积可见一斑。
默认情况下,编译过程中如果遇到宏包未安装而报错的情况下,Mik\TeX\ 会弹出一个对话框,让用户可以选择临时安装宏包,安装成功后继续编译。

\subsection{手动安装宏包}\label{subsec:pkg-manual-install}

\textbf{\textcolor{red}{如非万不得已,尽量不要手动安装宏包}}。绝大多数宏包都已打包到 \TeX\ Live 和 Mik\TeX\ 两大发行版的安装源,
可用宏包管理器安装。如果你知道某个宏包的名称,但不确定是否在发行版中已打包,可在 CTAN 中搜索。

如果确实有手动安装宏包的需要,本小节的内容将有所帮助。在手动安装之前,有必要了解一下 \TeX\ 目录结构(\TeX\ Directory Structure, TDS)。
它是 \TeX\ 发行版中宏包、字体、帮助文档等文件的组织结构。TDS 有时也称为 TEXMF 树,取 \TeX$+$\hologo{METAFONT} 之意。

以 \TeX\ Live 为例,系统的 TEXMF 树根目录为 \nolinkurl{C:\\texlive\\2015\\texmf-dist},其下有很多子目录,仅举几例:
\begin{description}
  \item[\texttt{tex/latex}] \LaTeX\ 宏包。
  \item[\texttt{doc/latex}] \LaTeX\ 宏包的帮助文档。
  \item[\texttt{source/latex}] \LaTeX\ 宏包的源代码。
  \item[\texttt{fonts/tfm}] \TeX\ 使用的字体文件,TFM 格式。
  \item[\texttt{fonts/type1}] PostScript 字体文件(Type1),PFB 格式。
  \item[\texttt{fonts/opentype}] OpenType 格式的字体文件。
\end{description}

需要手动安装的宏包,一般已经按照上述目录结构打包完成。手动安装时,尽量不要拷贝到系统的 TEXMF 树,而是拷贝到发行版提供的用户 TEXMF 树,如
\TeX Live 的 \nolinkurl{C:\\texlive\\texmf-local}。安装完成后,还需\textbf{刷新 \TeX\ 发行版的文件名数据库},令新安装的宏包文件能够被系统找到。
\TeX\ Live 用户须在 Windows 命令行或者 Linux 终端执行命令:
\begin{verbatim}
mktexlsr
\end{verbatim}
Mik\TeX\ 用户的命令为:
\begin{verbatim}
initexmf --update-fndb
\end{verbatim}

\endinput