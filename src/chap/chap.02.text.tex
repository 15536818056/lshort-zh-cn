\chapter{用 \LaTeX\ 排版文字}\label{chap:text}
\addtocontents{los}{\protect\addvspace{10pt}}

\begin{intro}
文字是排版的基础。在介绍 \LaTeX\ 支持的各种文档元素之前,首先应当了解一下如何在 \LaTeX\ 源代码中输入并排版文字,
尤其是(西文中的)标点符号、连字符、重音等一些细节。

本章还介绍了 \LaTeX\ 中排版中文的基本方式。随着 \LaTeX\ 和底层 \TeX\ 引擎的发展,旧方式(CCT、\pkg{CJK} 等)日渐退出舞台,
\texttt{xelatex} 和 \texttt{lualatex} 编译命令配合 \cls{ctex} 宏包 / 文档类的方式为当前的主流中文排版支持方式。
\end{intro}

\section{语言文字和编码}\label{sec:encoding}

\pinyinindex{bianma}{编码}
\LaTeX\ 源代码为文本文件,而文本文件的一个至关重要的性质是它的编码。在此用尽量短的篇幅介绍一下。

\subsection{ASCII 编码}\label{subsec:ascii}

\index{bianma@编码!ASCII}
计算机的基本存储单位是字节(byte),每个字节为八位(8-bit),范围用十六进制写作 \texttt{0x00}--\texttt{0xFF}。
ASCII (美国通用信息交换码)使用 \texttt{0x00}--\texttt{0x7F} 对文字编码,也就是 7-bit,能够表示一般
QWERTY 键盘上可输入的拉丁字母、数字和符号,以及一些不可显示的控制字符(如换行符、制表符等)。

由于 \TeX\ 最初设计用于排版以英文为主的西文文档,ASCII 编码完全够用,因而早期版本的 \TeX\ 只支持 7-bit 和 ASCII 编码。
要排版西欧语系中带重音的字符,就必须用到后文所叙述的重音命令,如 M\"obius 必须用 \verb|M\"obius| 输入。

\subsection{扩展编码}\label{subsec:ext-encoding}

在 ASCII 之后,各种语言文字都发展了自己的编码,比如西欧语言的 Latin-1、日本的 Shift-JIS、中国大陆的 GB2312-80 和 GBK 等。
它们中的绝大多数都向下兼容 ASCII,因此无论是在哪种编码下,\TeX\ 以及 \LaTeX\ 的命令和符号都能用。

\index{bianma@编码!Latin-1}
\pkgindex{inputenc}
\TeX\ 从 3.0 版开始支持 8-bit,能够处理源代码里编码处于 \texttt{0x80}--\texttt{0xFF} 范围的字符。
西欧(拉丁字母)、俄语系(西里尔字母)等语言的编码方案刚好能够利用 \texttt{0x80}--\texttt{0xFF} 这个范围,
处理起来较为方便。使用 \texttt{latex} 及 \texttt{pdflatex} 编译命令时,这些编码的处理由 \pkg{inputenc} 宏包支持。
比如将源代码相应编码,并在导言区调用 \pkg{inputenc} 宏包后,M\"obius 这样的词语就可以直接通过(用适当输入法)
输入 \texttt{M\"obius} 得到了%
\footnote{并非所有拉丁文扩展字符都能够直接使用 \pkg{inputenc} 宏包排版出来。一些符号需要借助 \pkg{fontenc} 宏包对字体做进一步配置。}。

\index{bianma@编码!GBK}
用于汉字的 GBK 等编码是多字节编码,ASCII 字符为一个字节,汉字等非 ASCII 字符为两个字节,使用
\texttt{latex} 及 \texttt{pdflatex} 编译命令时需要借助一些宏包进行较为复杂的判断和处理。早期,
\pkg{CJK} 宏包就是用于处理中、日、韩等多字节编码的语言文字的宏包。但 \pkg{CJK} 宏包的使用非常不方便,笔者不再推荐直接使用。

\subsection{UTF-8 编码}\label{subsec:utf8}

\index{bianma@编码!UTF-8}
Unicode 是一个多国字符的集合,覆盖了几乎全球范围内的语言文字。UTF-8 是 Unicode 的一套编码方案,
一个字符由一个到四个字节编码,其中单字节字符的编码与 ASCII 编码兼容。

现行版本的 \LaTeX\ 使用 UTF-8 作为默认编码%
\footnote{在 2018 年 4 月之前,需要调用 \pkg{inputenc} 宏包并指定 \texttt{utf8} 选项才能使用 UTF-8 编码。}。
因此,大部分西欧语言(拉丁字母)的文档可以在保存成 UTF-8 编码后直接编译:

\begin{verbatim}
\documentclass{article}
\begin{document}
Français Português Español Føroyskt
\end{document}
\end{verbatim}

由于种种限制,俄文(西里尔字母)、希腊文、阿拉伯文等不使用拉丁字母的文字系统,以及超过 8-bit 范围的 CJK 文字,
仍然不能按此种方法直接在 \LaTeX\ 中使用。

\pkgindex{inputenc,fontspec,babel,polyglossia}
较为现代的 \TeX\ 引擎,如 \hologo{XeTeX} 和 \hologo{LuaTeX},它们均原生支持 UTF-8 编码。
使用 \texttt{xelatex} 和 \texttt{lualatex} 排版时,将 \texttt{.tex} 源代码保存为 UTF-8 编码,并借助
\pkg{fontspec} 宏包(见 \ref{subsec:fontspec} 小节)调用适当的字体,原则上就可以在源代码中输入任意语言的文字。
注意此时\textbf{不再适用 \pkg{inputenc} 宏包}。

当然,良好的排版远不只是让字符能够显示出来,还要求断词规则、文字方向、标点禁则等诸多细节。因此,复杂语言
(如印地语、阿拉伯语等)的特殊排版还需要更多的宏包支持,如 \pkg{babel}、 \pkg{polyglossia} 等,此处不再涉及。

\section{排版中文}\label{sec:chinese}

用 \LaTeX\ 排版中文的一大难题是中文字体的处理。\TeX\ 使用的字体格式仅支持不超过 256 个字符,
旧式的中文排版方案如 \pkg{CJK} 宏包往往需要复杂的预处理,将中文字体拆分成数百个小字体,非常麻烦%
\footnote{现在的 \texttt{pdflatex} 以及 \texttt{latex} + \texttt{dvipdfmx} 命令通过字体映射的方式支持
直接使用中文字体,不需要经过预处理,但仍然是不方便新手操作的。}。
除此之外,中文排版中的一些细节,比如中英文之间插入间距、汉字之间允许断行、移除换行符带来的额外空白、
标点符号禁则处理(如句号、逗号不允许出现在行首)等都需要复杂而繁琐的设置。

\pkgindex{CJK,xeCJK,luatexja}
为了解决上述问题,在不同的 \TeX\ 引擎下需要使用不同的宏包,如 \texttt{latex} 和 \texttt{pdflatex} 中需要 \pkg{CJK} 包,
\texttt{xelatex} 中需要 \pkg{xeCJK} 包,\texttt{lualatex} 中需要 \pkg{luatexja} 包。对一般用户而言,仍比较麻烦。

\pkgindex{ctex}
\clsindex{ctexart,ctexbook,ctexrep}
\pkg{ctex} 宏包和文档类对 \pkg{CJK}、\pkg{xeCJK}、\pkg{luatexja} 等宏包的进一步封装%
\footnote{C\TeX\ 还经常用来指一个过时的 \TeX\ 发行版,注意与这里的 \pkg{ctex} 宏包和文档类区分。}。
\pkg{ctex} 文档类包括\cls{ctexart} / \cls{ctexrep} / \cls{ctexbook},对应 \LaTeX\ 的三个标准文档类%
\footnote{\pkg{ctex} 文档类中还包含有 \cls{ctexbeamer},对应原 \cls{beamer} 文档类,用于排版幻灯片。}。
它们对 \LaTeX\ 默认的排版样式做了一些调整,以切合中文排版风格。\pkg{ctex} 宏包 / 文档类还能够根据
操作系统自动选择字体,因此基本无需额外配置即可排版中文文档。下面举一个简单的例子:
\begin{verbatim}
\documentclass{ctexart}
\begin{document}
中文 \LaTeX{} 排版。
汉字 English 汉字English之间手动输入的空格不会影响
排版结果。换行符也不会影响。
\end{document}
\end{verbatim}

\index{xelatex@\texttt{xelatex} 命令}
\index{lualatex@\texttt{lualatex} 命令}
注意源代码须保存为 UTF-8 编码,并使用 \texttt{xelatex} 或 \texttt{lualatex} 命令编译。

虽然 \pkg{ctex} 宏包和文档类保留了对 GBK 编码以及 \texttt{latex} + \texttt{dvipdfmx}、\texttt{pdflatex} 等旧的编译命令的支持,
笔者建议在使用 \pkg{ctex} 宏包和文档类时总是将源代码保存为 UTF-8 编码,用 \texttt{xelatex} 命令编译。

\section{\LaTeX\ 中的字符}\label{sec:text-symbols}

\subsection{空格和分段}\label{subsec:spaces}

\LaTeX\ 源代码中,空格键和 Tab 键输入的空白字符视为“空格”。
连续的若干个空白字符视为一个空格。一行开头的空格忽略不计。

\cmdindex{par}
行末的换行符视为一个空格;但连续两个换行符,也就是空行,会将文字分段。多个空行被视为一个空行。
也可以在行末使用 \cmd{par} 命令分段。
\begin{example}
Several spaces     equal one.
  Front spaces are ignored.

An empty line starts a new
paragraph.\par
A \verb|\par| command does
the same.
\end{example}

\subsection{注释}\label{subsec:comments}

\index{^^e@\texttt\% (\textit{注释})}
\LaTeX\ 用 \texttt\% 字符作为注释。在这个字符之后直到行末,所有的字符都被忽略,
行末的换行符也不引入空格。
\begin{example}
This is an % short comment
% ---
% Long and organized
% comments
% ---
example: Comments do not bre%
ak a word.
\end{example}

\subsection{特殊字符}\label{subsec:special-chars}

以下字符在 \LaTeX\ 里有特殊用途,如 \texttt\% 表示注释, \texttt\$、\texttt\textasciicircum 、\texttt\_ 等用于排版数学公式,
\texttt\& 用于排版表格,等等。直接输入这些字符得不到对应的符号,还往往会出错:
\begin{verbatim}
# $ % & { } _ ^ ~ \
\end{verbatim}

如果想要输入以上符号,需要使用以下带反斜线的形式输入:
\begin{example}
\# \$ \% \& \{ \} \_
\^{} \~{} \textbackslash
\end{example}

事实上这些带反斜线的形式就是 \LaTeX\ 命令。\cmd{\textasciicircum} 和 \cmd{\textasciitilde}
两个命令是需要带参数的,如果不加一对花括号(空参数),就将后面的字符作为参数,形成重音效果(详见 \ref{subsec:accents} 节)。
\crcmd\ 被直接定义成了手动换行的命令,输入反斜杠就只好用 \cmd{text\-back\-slash}。

\subsection{连字}\label{subsec:ligatures}

西文排版中经常会出现连字(Ligatures),常见的有 ff / fi / fl / ffi / ffl{}。
\begin{example}
It's difficult to find \ldots .\\
It's dif{}f{}icult to f{}ind \ldots .
\end{example}

\subsection{标点符号}\label{subsec:punct}

中文的标点符号(绝大多数为非 ASCII 字符)使用中文输入法输入即可,一般不需要过多留意。
而在英文标点符号上,有许多需要留意的地方。

\subsubsection{引号}

\index{biaodianfuhao@标点符号!`@\protect\verb+`+ (\textit{前引号} `)}
\index{biaodianfuhao@标点符号!``@\protect\verb+'+ (\textit{后引号} ')}
\index{biaodianfuhao@标点符号!'@\protect\verb+``+ (\textit{前双引号} ``)}
\index{biaodianfuhao@标点符号!''@\protect\verb+''+ (\textit{后双引号} '')}
\LaTeX\ 的单引号 ` ' 用 \texttt` 和 \texttt' 输入;双引号 `` 和 '' 分别用 \verb|``| 和 \verb|''| 输入
(虽然 \texttt" 也表示双引号,但没有合适的单个符号用来表示前双引号,于是习惯上写为与 \verb|``| 对应的 \verb|''|)。
\begin{example}
``Please press the `x' key.''
\end{example}

\subsubsection{连字号和破折号}

\index{biaodianfuhao@标点符号!-@\protect\verb+-+ (\textit{连字号} -)}
\index{biaodianfuhao@标点符号!--@\protect\verb+--+ (\textit{短破折号} --)}
\index{biaodianfuhao@标点符号!---@\protect\verb+---+ (\textit{长破折号} ---)}
\LaTeX\ 中有三种长度的“横线”可用:连字号(hyphen)、短破折号(en-dash)和长破折号(em-dash)。
它们分别有不同的用途:连字号 - 用来组成复合词;短破折号 -- 用来连接数字表示范围;长破折号 --- 用来连接单词,与中文标点符号中的破折号用法类似。
\begin{example}
daughter-in-law, X-rated\\
pages 13--67\\
yes---or no?
\end{example}

\subsubsection{省略号}

\index{biaodianfuhao@标点符号!ldots@\cmd{ldots} (\textit{省略号} \ldots)}
\index{biaodianfuhao@标点符号!dots@\cmd{dots} (\textit{省略号} \dots)|see{\cmd{ldots}}}
\LaTeX\ 提供了命令 \cmd{ldots} 来生成省略号,相对于直接输入三个点的方式更为合理。
\cmd{ldots} 和 \cmd{dots} 是两个等效的命令。
\begin{example}
one, two, three, \ldots one hundred.
\end{example}

\subsubsection{波浪号}

我们在 \ref{subsec:special-chars} 小节中了解了 \cmd{\textasciitilde} 命令,它可以用来输入波浪号,
但位置靠顶端(\cmd{\textasciitilde} 命令主要用作重音,参考下一小节)。西文中较少将波浪号作为标点符号使用,通常使用中文的全角波浪号(~)。

\subsection{拉丁文扩展与重音}\label{subsec:accents}

\pinyinindex{zhongyin}{重音}
\LaTeX\ 支持用命令输入各种西欧语言所使用的拉丁文扩展字母,主要为带重音的字母:
\begin{example}
H\^otel, na\"\i ve, \'el\`eve,\\
sm\o rrebr\o d, !`Se\ norita!,\\
Sch\"onbrunner Schlo\ss{}
Stra\ss e
\end{example}

更多可用的符号和重音见表 \ref{tbl:accents}。注意与 \ref{subsec:math-accents} 小节的数学重音区分开来。

\def\TSYM #1{#1       & \texttt{\string#1}}
\def\TACC #1#2{#1{#2} & \texttt{\string#1#2}}       % accents using a control character
\def\TTACC #1#2{#1{#2} & \texttt{\string#1 #2}}     % accents using a control word
\def\WTACC #1#2{#1{#2} & \texttt{\string#1\{#2\}}}  % multi-letter accents
\begin{table}[htp]
\centering
\caption{\LaTeX\ 文本中的重音和特殊字符。} \label{tbl:accents}
\begin{tabular}{*4{cl}}
 \hline
 \TACC{\`}{o} & \TACC{\'}{o} & \TACC{\^}{o} & \TACC{\~}{o} \\
 \TACC{\=}{o} & \TACC{\.}{o} & \TACC{\"}{o} & \TTACC{\r}{o}\\
 \TTACC{\u}{o} & \TTACC{\v}{o} & \TTACC{\H}{o} & \TTACC{\c}{o} \\
 \TTACC{\d}{o} & \TTACC{\b}{o} & \WTACC{\t}{oo} \\[6pt]
 \TSYM{\oe} & \TSYM{\OE} & \TSYM{\ae} & \TSYM{\AE} \\
 \TSYM{\aa} & \TSYM{\AA} & \TSYM{\ss} \\[6pt]
 \TSYM{\o}  & \TSYM{\O}  & \TSYM{\l} & \TSYM{\L} \\
 \TSYM{\i}  & \TSYM{\j}  & !` & \verb|!`| & ?` & \verb|?`| \\
 \hline
\end{tabular}
\begin{quote}\footnotesize%
前四行实际上都是带一个参数的命令。\cmd{\textasciicircum}\texttt{o} 也可以写作
\cmd{\textasciicircum}\marg*{o},以此类推。
\end{quote}
\end{table}

\subsection{其它符号}\label{subsec:text-misc}

\symindex{dag,ddag,P,S,copyright,pounds}
\symindex{textasteriskcentered,textperiodcentered,textbullet,textregistered,texttrademark}
\LaTeX\ 预定义了其它一些文本模式的符号,部分符号可参考表 \ref{tbl:general-syms}。
\begin{example}
\P{} \S{} \dag{} \ddag{}
\copyright{} \pounds{}

\textasteriskcentered
\textperiodcentered
\textbullet

\textregistered{} \texttrademark
\end{example}

更多的符号多由特定的宏包支持。参考文献 \cite{symbols} 搜集了所有在 \TeX\ 发行版中可用的符号,使用时要留意每个符号所依赖的宏包。

\subsection{\LaTeX\ 标志}\label{subsec:latex-mark}

\cmdindex{TeX,LaTeX,LaTeXe}
我们见到的所有错落有致的 \LaTeX\ 标志都是由以下命令输入的:
\begin{center}
\begin{tabular}{*{2}{l}}
 \hline
 \TeX & \cmd{TeX} \\
 \LaTeX & \cmd{LaTeX} \\
 \LaTeXe & \cmd{LaTeXe} \\
 \hline
\end{tabular}
\end{center}

\section{文字强调}\label{sec:emph}

强调文字的方法,要么是添加下划线等装饰物,要么是改变文字的字体。

\cmdindex{underline}
\LaTeX\ 定义了 \cmd{underline} 命令用来为文字添加下划线:
\begin{example}
An \underline{underlined} text.
\end{example}

\pkgindex{ulem}
\cmdindex[ulem]{uline}
\cmd{underline} 命令生成下划线的样式比较机械,不同的单词可能生成高低各异的下划线,并且无法换行。
\pkg{ulem} 宏包解决了这一问题,它提供的 \cmd{uline} 命令能够轻松生成自动换行的下划线:
\begin{example}
An example of \uline{some
long and underlined words.}
\end{example}

\cmdindex{emph}
\cmd{emph} 命令用来将文字变为斜体以示强调。如果在本身已经用 \cmd{emph} 命令强调的文字内部嵌套使用 \cmd{emph} 命令,
内部则使用直立体文字:
\begin{example}
Some \emph{emphasized words,
including \emph{double-emphasized}
words}, are shown here.
\end{example}

如果需要用其它字体表示强调,可参考后文的 \ref{subsec:fontshape} 小节。

\section{断行和断页}\label{sec:break}

\LaTeX\ 将文字段落在合适的位置进行断行,尽可能做到每行的宽度一致,并且单词间距匀称。
文字段落和公式、图表等内容从上到下顺序排布,并在合适的位置断页,分割成匀称的页面。
在绝大多数时候,我们无需自己操心断行和断页。但偶尔会遇到需要手工调整的地方。

\subsection{单词间距}\label{subsec:interword}

在西文排版实践中,断行的位置尽可能选取在两个单词之间,也就是用户在源文件中输入的“空格”%
\footnote{中文排版实现汉字间断行,则需要宏包(如 \pkg{xeCJK} 等)或特殊的排版引擎(如 up\LaTeX{})的支持。}。
这个“空格”通常生成一个间距,它会根据行宽和上下文自动调整,文字密一些的地方,单词间距就略窄,反之略宽。

\index{~@\texttt\textasciitilde\ (\textit{不断行空格})}
\LaTeX\ 按照西文排版习惯在单词间的“空格”处断行,“空格”生成的间距也随之舍去。
我们可以使用字符 \texttt\textasciitilde 在合适的位置插入一个不会断行的空格(高德纳称之为 tie,“带子”),
通常用在英文人名、图表名称等场景:
\begin{example}
Fig.~2a \\
Donale~E. Knuth
\end{example}

\subsection{手动断行和断页}\label{subsec:manual-break}

\index{\@\crcmd\ (\textit{换行})} \cmdindex{newline}
如果我们确实需要手动断行,可使用如下命令:
\begin{command}
\crcmd \oarg{length} \\
\crcmd*\oarg{length} \\
\cmd{newline}
\end{command}

它们有两点区别:一是 \crcmd\ 可以带可选参数 \Arg{length},用于在断行处向下增加垂直间距(见 \ref{subsec:vspace} 小节),
而 \cmd{newline} 不带可选参数;二是 \crcmd\ 也在表格、公式等地方用于换行,而 \cmd{newline} 只用于文本段落中。
带星号的 \crcmd\ 使得断行处禁止分页。

\begin{example}
另外需要注意的是,使用 \verb|\\|
断行命令 \\ 不会令内容另起一段,
而是在段落中直接开始新的一行。
\end{example}

\cmdindex{newpage,clearpage}
断页的命令有两个:
\begin{command}
\cmd{newpage} \\
\cmd{clearpage}
\end{command}

通常情况下两个命令都能起到另起一页的作用,但有一些区别:一是在双栏排版模式中 \cmd{newpage} 只起到另起一栏的作用;二是涉及到浮动体的排版上行为不同。
后文的 \ref{sec:float} 节以及 \ref{subsec:columns} 小节会更详细地介绍相关内容。

\cmdindex{linebreak,nolinebreak}
有时候我们不满足于 \LaTeX\ 默认的断行和断页位置,需要进行微调,可以用以下命令告诉 \LaTeX\ 哪些地方适合断行或断页,哪些地方不适合:
\begin{command}
\cmd{linebreak}\oarg{n} \quad \cmd{nolinebreak}\oarg{n} \\
\cmd{pagebreak}\oarg{n} \quad \cmd{nopagebreak}\oarg{n}
\end{command}

以上命令都带一个可选参数,用数字 \Arg{n} 代表适合/不适合的程度,取值范围为 0-4,不带可选参数时,缺省为 4。
比如 \cmd{linebreak}\texttt{[4]} 意味着此处需要强行断行;\cmd{nopagebreak} 或 \cmd{nopagebreak}\texttt{[4]}
意味着禁止在此处断页。

以上命令适合给出优先考虑断行断页/禁止断行断页的位置,但不适合直接拿来断行或断页,使用 \cmd{newline} 或 \cmd{newpage} 等是更好的选择。
因为 \cmd{newline} 和 \cmd{newpage} 会在断行/断页位置填充适当的间距,但 \cmd{linebreak} 和 \cmd{pagebreak} 不能,
它们的强行断行/断页可能会制造出糟糕的排版效果,并导致 \LaTeX\ 报 \texttt{Underfull \cmd{hbox}} 等警告。

\begin{example}
使用 \verb|\newline| 断行的效果
\newline
与使用 \verb|\linebreak| 断行的效果
\linebreak
进行对比。
\end{example}

\subsection{断词}\label{subsec:hyphen}

如果 \LaTeX\ 遇到了很长的英文单词,仅在单词之间的“空格”处断行无法生成宽度匀称的行时,就会考虑从单词中间断开。
对于绝大部分单词,\LaTeX\ 能够找到合适的断词位置,在断开的行尾加上连字符 \char\hyphenchar\font 。

\index{-@\cmd{-} (\textit{断词})}
如果一些单词没能自动断词,我们可以在单词内手动使用 \cmd{-} 命令指定断词的位置:
\begin{example}
I think this is: su\-per\-cal\-%
i\-frag\-i\-lis\-tic\-ex\-pi\-%
al\-i\-do\-cious.
\end{example}

\endinput
