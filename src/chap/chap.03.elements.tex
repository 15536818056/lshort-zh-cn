\chapter{文档元素}\label{chap:elements}
\addtocontents{los}{\protect\addvspace{10pt}}

\begin{intro}
在知道了如何输入文字后,我们将在本章了解一个结构化的文档所依赖的各种元素——章节、目录、列表、图表、交叉引用、脚注等等。
\end{intro}

\section{章节和目录}\label{sec:secs}

\subsection{章节标题}\label{subsec:secs}

\cmdindex{part,chapter,section,subsection,subsubsection,paragraph,subparagraph}
\clsindex{article,book,report}

一篇结构化的、条理清晰文档一定是层次分明的,通过不同的命令分割为章、节、小节。三个标准文档类 \cls{article}、\cls{report} 和 \cls{book}%
\footnote{千万注意是\textbf{标准文档类},其它文档类,如果不是从标准文档类衍生而来,
很可能没有定义或只定义了一部分命令,如 \cls{beamer} 或 \cls{moderncv} 等。}%
提供了划分章节的命令:
\begin{command}
\cmd{chapter}\marg{title} \quad
\cmd{section}\marg{title} \quad
\cmd{subsection}\marg{title} \\
\cmd{subsubsection}\marg{title} \quad
\cmd{paragraph}\marg{title} \quad
\cmd{subparagraph}\marg{title}
\end{command}
其中 \cmd{chapter} \textbf{只在 \cls{book} 和 \cls{report} 文档类有定义}。这些命令生成章节标题,并能够自动编号。
除此之外 \LaTeX{} 还提供了 \cmd{part} 命令,用来将整个文档分割为大的分块,但不影响 \cmd{chapter} 或 \cmd{section} 等的编号。

上述命令除了生成带编号的标题之外,还向目录中添加条目,并影响页眉页脚的内容(详见 \ref{sec:pagestyle} 节)。每个命令有两种变体:
\begin{itemize}
  \item 带可选参数的变体:\cmd{section}\oarg{short title}\marg{title}\par
  标题使用 \Arg{title} 参数,在目录和页眉页脚中使用 \Arg{short title} 参数;
  \item 带星号的变体:\cmd{section*}\marg{title}\par
  标题不带编号,也不生成目录项和页眉页脚。
\end{itemize}

较低层次如 \cmd{paragraph} 和 \cmd{subparagraph} 即使不用带星号的变体,生成的标题默认也不带编号,事实上,除 \cmd{part} 外:
\begin{itemize}
  \item \cls{article} 文档类带编号的层级为 \cmd{section} / \cmd{subsection} / \cmd{sub\-sub\-section} 三级;
  \item \cls{report}/\cls{book} 文档类带编号的层级为 \cmd{chapter} / \cmd{section} / \cmd{sub\-section} 三级。
\end{itemize}
对此的详细解释和调整方法见 \ref{subsec:latex-counts} 小节。

\pkgindex{titlesec}
\LaTeX{} 及标准文档类并未提供为 \cmd{section} 等章节命令定制格式的功能,这一功能由 \pkg{titlesec} 宏包提供。详情请参考宏包的帮助文档。

\subsection{目录}\label{sec:toc}

\cmdindex{tableofcontents}
\pkgindex{tocbibind,tocloft,titletoc}
在 \LaTeX{} 中生成目录非常容易,只需在合适的地方使用命令:
\begin{command}
\cmd{tableofcontents}
\end{command}

这个命令会生成单独的一章(\cls{book} / \cls{report})或一节(\cls{article}),标题默认为 ``Contents''{},可通过 \ref{sec:latex-settings} 节给出的方法定制标题。
\cmd{tableof\-contents} 生成的章节默认不写入目录(\cmd{section*} 或 \cmd{chapter*}),可使用 \pkg{tocbibind} 等宏包修改设置。

正确生成目录项,一般需要编译两次源代码。

\cmdindex{addcontentsline}
有时我们使用了 \cmd{chapter*} 或 \cmd{section*} 这样不生成目录项的章节标题命令,
而又想手动生成该章节的目录项,可以在标题命令后面使用:
\begin{command}
\cmd{addcontentsline}\marg*{toc}\marg{level}\marg{title}
\end{command}

其中 \Arg{level} 为章节层次 \texttt{chapter} 或 \texttt{section} 等,\Arg{title} 为出现于目录项的章节标题。

\pkg{titletoc}、\pkg{tocloft} 等宏包提供了具体定制目录项格式的功能,详情请参考宏包的帮助文档。

\subsection{文档结构的划分}\label{sec:matters}

\cmdindex{appendix}
所有标准文档类都提供了一个 \cmd{appendix} 命令将正文和附录分开%
\footnote{有的参考文档可能使用 \cmd{begin} \marg*{appendix} \ldots \cmd{end} \marg*{appendix} 这样的写法,
虽然有效,但并不规范,只要使用 \cmd{appendix} 命令就够了。},
使用 \cmd{appendix} 后,最高一级章节改为使用拉丁字母编号,从 A 开始。

\clsindex{book}
\cmdindex{frontmatter,mainmatter,backmatter}
\cls{book} 文档类还提供了前言、正文、后记结构的划分命令:
\begin{description}
  \item[\cmd{frontmatter}] 前言部分,页码为小写罗马字母格式;其后的 \cmd{chapter} 不编号。
  \item[\cmd{mainmatter}] 正文部分,页码为阿拉伯数字格式,从 1 开始计数;其后的章节编号正常。
  \item[\cmd{backmatter}] 后记部分,页码格式不变,继续正常计数;其后的 \cmd{chapter} 不编号。
\end{description}

\cmdindex{include}
以上三个命令还可和 \cmd{appendix} 命令结合,生成有前言、正文、附录、后记四部分的文档。
源代码 \ref{code:book-struct} 结合 \ref{sec:latex-multi-files} 节的 \cmd{include} 命令和其它一些命令示意了一份完整的文档结构。

\begin{sourcecode}[htp]
\begin{Verbatim}
\documentclass[...]{book}
% 导言区,加载宏包和各项设置,包括参考文献、索引等
\usepackage{...}
\usepackage{makeidx}
\makeindex
\bibliographystyle{...}

\begin{document}
\frontmatter
\maketitle % 标题页
\include{preface} % 前言章节 preface.tex
\tableofcontents
\mainmatter
\include{chapter1} % 第一章 chapter1.tex
\include{chapter2} % 第二章 chapter2.tex
...
\appendix
\include{appendixA} % 附录 A appendixA.tex
...
\backmatter
\include{prologue} % 后记 prologue.tex
\bibliography{...} % 利用 BibTeX 工具生成参考文献
\printindex        % 利用 makeindex 工具生成索引
\end{document}
\end{Verbatim}
\caption{\cls{book} 文档类的文档结构示例。}\label{code:book-struct}
\end{sourcecode}

\section{标题页}\label{sec:titlepage}

\cmdindex{title,author,date,today}
\cmdindex{thanks,and}
\LaTeX{} 支持生成简单的标题页。首先需要给定标题和作者等信息:
\begin{command}
\cmd{title}\marg{title} \quad
\cmd{author}\marg{author} \quad
\cmd{date}\marg{date}
\end{command}
其中前两个命令是必须的(不用 \cmd{title} 会报错;不用 \cmd{author} 会警告),\cmd{date} 命令可选。
\LaTeX{} 还提供了一个 \cmd{today} 命令自动生成当前日期,\cmd{date} 默认使用 \cmd{today}。
在 \cmd{title}、\cmd{author} 等命令内可以使用 \cmd{thanks} 命令生成标题页的脚注,用 \cmd{and} 隔开多个人名。

\cmdindex{maketitle}
\pinyinindex{biaotiye}{标题页}
在信息给定后,就可以使用 \cmd{maketitle} 命令生成一个简单的标题页了。
源代码 \ref{code:titlepage} 给出了一个标题页的示例和大致效果。
\cls{article} 文档类的标题默认不单独成页,而 \cls{report} 和 \cls{book} 默认单独成页。
可在 \cmd{document\-class} 命令调用文档类时指定 \texttt{titlepage / notitlepage} 选项以修改默认的行为。

\begin{sourcecode}[htp]
\begin{Verbatim}
\title{Test title}
\author{ Mary\thanks{E-mail:*****@***.com}
  \and Ted\thanks{Corresponding author}
  \and Louis}
\date{\today}
\end{Verbatim}
\bigskip
\makeatletter
\begin{minipage}{\textwidth}
    \centering
    \renewcommand\thempfootnote{\@fnsymbol\c@mpfootnote}%
    {\LARGE Test title \par}%
    \vskip 1.5em%
    {\large
        \lineskip .5em%
        \begin{tabular}[t]{c}%
            Mary\footnote{E-mail:*****@***.com}
        \end{tabular}
        \hskip 1em \@plus.17fil%
        \begin{tabular}[t]{c}%
            Ted\footnote{Corresponding author}
        \end{tabular}
        \hskip 1em \@plus.17fil%
        \begin{tabular}[t]{c}%
            Louis
        \end{tabular}\par}%
    \vskip 1em%
    {\large September 10, 2015}%
\end{minipage}
\makeatother
\caption{\LaTeX{} 默认的标题页示例和效果。}\label{code:titlepage}
\end{sourcecode}

\envindex{titlepage}
\LaTeX{} 标准类还提供了一个简单的 \env{titlepage} 环境,生成不带页眉页脚的一页。用户可以在这个环境中使用各种排版元素自由发挥,
生成自定义的标题页以替代 \cmd{maketitle} 命令。甚至可以利用 \env{titlepage} 环境重新定义 \cmd{maketitle}:
\begin{verbatim}
\renewcommand{\maketitle}{\begin{titlepage}
... % 用户自定义命令
\end{titlepage}}
\end{verbatim}

事实上,为标准文档类指定了 \texttt{titlepage} 选项以后,使用 \cmd{maketitle} 命令生成的标题页就是一个 \env{titlepage} 环境。

以上是 \LaTeX{} 标准文档类的标题页相关命令用法。在各种文档模板中经常有自定义的标题页,
有可能需要除了 \cmd{title} 和 \cmd{author} 以外的命令给定信息,用法也可能与标准文档类的不一致
(甚至有些模板可能没有定义 \env{titlepage} 等环境)。使用文档模板前\textbf{一定要仔细阅读文档模板的帮助文档}。

\section{交叉引用}\label{sec:crossref}

\cmdindex{label}
交叉引用是 \LaTeX{} 强大的自动排版功能的体现之一。在能够被交叉引用的地方,如章节、公式、图表、定理等位置使用 \cmd{label} 命令:
\begin{command}
\cmd{label}\marg{label-name}
\end{command}

\cmdindex{ref,pageref}
之后可以在别处使用 \cmd{ref} 或 \cmd{pageref} 命令,分别生成交叉引用的编号和页码:
\begin{command}
\cmd{ref}\marg{label-name} \quad
\cmd{pageref}\marg{label-name}
\end{command}
\begin{example}
A reference to this subsection
\label{sec:this} looks like:
``see section~\ref{sec:this} on
page~\pageref{sec:this}.''
\end{example}

为了生成正确的交叉引用,一般也需要多次编译源代码。

\cmd{label} 命令可用于记录各种类型的交叉引用,使用位置分别为:
\begin{description}
  \item[章节标题] 在章节标题命令 \cmd{section} 等之后紧接着使用。
  \item[行间公式] 单行公式在公式内任意位置使用;多行公式在每一行公式的任意位置使用。
  \item[有序列表] 在 \env{enumerate} 环境的每个 \cmd{item} 命令之后、下一个 \cmd{item} 命令之前任意位置使用。
  \item[图表标题] 在图表标题命令 \cmd{caption} 之后紧接着使用。
  \item[定理环境] 在定理环境内部任意位置使用。
\end{description}

在使用不记编号的命令形式(\cmd{section*}、\cmd{caption*}、带可选参数的 \cmd{item} 命令等)时不要使用 \cmd{label} 命令,
否则生成的引用编号不正确。

\section{脚注和边注}\label{sec:footnote-marginpar}

\cmdindex{footnote}
使用 \cmd{footnote} 命令可以在页面底部生成一个脚注:
\begin{command}
\cmd{footnote}\marg{footnote}
\end{command}

假如我们输入以下文字和命令:
\begin{verbatim}
“天地玄黄,宇宙洪荒。日月盈昃,辰宿列张。”\footnote{出自《千字文》。}
\end{verbatim}

在正文中则为:%
“天地玄黄,宇宙洪荒。日月盈昃,辰宿列张。”\footnote{出自《千字文》。}

\cmdindex{footnotemark,footnotetext}
有些情况下(比如在表格环境、各种盒子内)使用 \cmd{footnote} 并不能正确生成脚注。我们可以分两步进行,
先使用 \cmd{foot\-note\-mark} 为脚注计数,再在合适的位置用 \cmd{foot\-note\-text} 生成脚注。比如:

\begin{verbatim}
\begin{tabular}{l}
\hline
“天地玄黄,宇宙洪荒。日月盈昃,辰宿列张。”\footnotemark \\
\hline
\end{tabular}
\footnotetext{表格里的名句出自《千字文》。}
\end{verbatim}

效果为:

\leavevmode\begin{tabular}{l}
\hline
“天地玄黄,宇宙洪荒。日月盈昃,辰宿列张。”\footnotemark \\
\hline
\end{tabular}
\footnotetext{表格里的名句出自《千字文》。}

\cmdindex{marginpar}
使用 \cmd{marginpar} 命令可在边栏位置生成边注:
\begin{command}
\cmd{marginpar}\oarg{left-margin}\marg{right-margin}
\end{command}
如果只给定了 \Arg{right-margin},那么边注在奇偶数页文字相同;如果同时给定了 \Arg{left-margin},
则偶数页使用 \Arg{left-margin} 的文字。

例如以下代码:
\begin{verbatim}
\marginpar{\footnotesize 边注较窄,不要写过多文字,最好设置较小的字号。}
\end{verbatim}
其效果见边栏。\marginpar{\footnotesize 边注较窄,不要写过多文字。最好设置较小的字号。}

\section{特殊环境}\label{sec:envs}

\subsection{列表}\label{subsec:lists}

\envindex{enumerate,itemize}
\cmdindex{item}
\LaTeX{} 提供了基本的有序和无序列表环境 \env{enumerate} 和 \env{itemize},两者的用法很类似,都用 \cmd{item} 标明每个列表项。
\env{enumerate} 环境会自动对列表项编号。
\begin{command}
\cmd{begin}\marg*{enumerate} \\
\cmd{item} \ldots \\
\cmd{end}\marg*{enumerate}
\end{command}

其中 \cmd{item} 可带一个可选参数,将有序列表的计数或者无序列表的符号替换成自定义的符号。
列表可以嵌套使用,最多嵌套四层。
\begin{example}
\begin{enumerate}
  \item An item.
  \begin{enumerate}
    \item A nested item.\label{itref}
    \item[*] A starred item.
  \end{enumerate}
  \item Reference(\ref{itref}).
\end{enumerate}
\end{example}

\begin{example}
\begin{itemize}
  \item An item.
  \begin{itemize}
    \item A nested item.
    \item[+] A `plus' item.
    \item Another item.
  \end{itemize}
  \item Go back to upper level.
\end{itemize}
\end{example}

\envindex{description}
关键字环境 \env{description} 的用法与以上两者类似,不同的是 \cmd{item} 后的可选参数用来写关键字,以粗体显示,一般是必填的:
\begin{command}
\cmd{begin}\marg*{description} \\
  \cmd{item}\oarg{item title} \ldots \\
\cmd{end}\marg*{description}
\end{command}

\begin{example}
\begin{description}
  \item[Enumerate] Numbered list.
  \item[Itemize] Non-numbered list.
\end{description}
\end{example}

各级无序列表的符号由命令 \cmd{labelitemi} 到 \cmd{labelitemiv} 定义,可以简单地重新定义它们:
\begin{example}
\renewcommand{\labelitemi}{\ddag}
\renewcommand{\labelitemii}{\dag}
\begin{itemize}
  \item First item
  \begin{itemize}
    \item Subitem
    \item Subitem
  \end{itemize}
  \item Second item
\end{itemize}
\end{example}

有序列表的符号由命令 \cmd{labelenumi} 到 \cmd{labelenumiv} 定义,
重新定义这些命令需要用到 \ref{sec:counters} 节的计数器相关命令:
\begin{example}
\renewcommand{\labelenumi}%
  {\Alph{enumi}>}
\begin{enumerate}
  \item First item
  \item Second item
\end{enumerate}
\end{example}

默认的列表间距比较宽,\LaTeX{} 本身也未提供方便的定制功能,可用 \pkg{enumitem} 宏包定制各种列表间距。
\pkg{enumitem} 宏包还提供了对列表标签、引用等的定制。有兴趣的读者可参考其帮助文档。

\subsection{对齐环境}\label{subsec:flush}

\envindex{center,flushleft,flushright}
\env{center}、\env{flush\-left} 和 \env{flush\-right} 环境分别用于生成居中、左对齐和右对齐的文本环境。
\begin{command}
\cmd{begin}\marg*{center} \ldots\ \cmd{end}\marg*{center} \\
\cmd{begin}\marg*{flushleft} \ldots\ \cmd{end}\marg*{flushleft} \\
\cmd{begin}\marg*{flushright} \ldots\ \cmd{end}\marg*{flushright}
\end{command}

\begin{example}
\begin{center}
Centered text using a
\verb|center| environment.
\end{center}
\begin{flushleft}
Left-aligned text using a
\verb|flushleft| environment.
\end{flushleft}
\begin{flushright}
Right-aligned text using a
\verb|flushright| environment.
\end{flushright}
\end{example}

\cmdindex{centering,raggedleft,raggedright}
除此之外,还可以用以下命令直接改变文字的对齐方式:
\begin{command}
\cmd{centering} \quad
\cmd{raggedright} \quad
\cmd{raggedleft}
\end{command}

\begin{example}
\centering
Centered text paragraph.

\raggedright
Left-aligned text paragraph.

\raggedleft
Right-aligned text paragraph.
\end{example}

三个命令和对应的环境经常被误用,有直接用所谓 \cmd{flushleft} 命令或者 \env{raggedright} 环境的,都是不甚严格的用法(即使它们可能有效)。
有一点可以将两者区分开来:\env{center} 等环境会在上下文产生一个额外间距,而 \cmd{centering} 等命令不产生,只是改变对齐方式。
比如在浮动体环境 \env{table} 或 \env{figure} 内实现居中对齐,用 \cmd{centering} 命令即可,没必要再用 \env{center} 环境。

\subsection{引用环境}\label{subsec:quote}

\envindex{quote,quotation}
\LaTeX{} 提供了两种引用的环境:\env{quote} 用于引用较短的文字,首行不缩进;\env{quotation} 用于引用若干段文字,首行缩进。
引用环境较一般文字有额外的左右缩进。
\begin{example}
Francis Bacon says:
\begin{quote}
Knowledge is power.
\end{quote}
\end{example}

\begin{example}
《木兰诗》:
\begin{quotation}
万里赴戎机,关山度若飞。
朔气传金柝,寒光照铁衣。
将军百战死,壮士十年归。

归来见天子,天子坐明堂。
策勋十二转,赏赐百千强。……
\end{quotation}
\end{example}

\envindex{verse}
\env{verse} 用于排版诗歌,与 \env{quotation} 恰好相反,\env{verse} 是首行悬挂缩进的。
\begin{example}
Rabindranath Tagore's short poem:
\begin{verse}
Beauty is truth's smile
when she beholds her own face in
a perfect mirror.
\end{verse}
\end{example}

\subsection{摘要环境}\label{subsec:abstract}

\envindex{abstract}
摘要环境 \env{abstract} 默认只在标准文档类中的 \cls{article} 和 \cls{report} 文档类可用,
一般用于紧跟 \cmd{maketitle} 命令之后介绍文档的摘要。如果文档类指定了 \texttt{titlepage} 选项,则单独成页;
反之,单栏排版时相当于一个居中的小标题加一个 \env{quotation} 环境,双栏排版时相当于 \cmd{section*} 定义的一节。

\subsection{代码环境}\label{subsec:verbatim}

\envindex{verbatim}
有时我们需要将一段代码原样转义输出,这就要用到代码环境 \env{verbatim},它以等宽字体排版代码,回车和空格也分别起到换行和空位的作用;
带星号的版本更进一步将空格显示成“\textvisiblespace”。
\begin{example}
\begin{verbatim}
#include <iostream>
int main()
{
  std::cout << "Hello, world!"
            << std::endl;
  return 0;
}
\end{verbatim}
\end{example}

\begin{example}
\begin{verbatim*}
for (int i=0; i<4; ++i)
  printf("Number %d\n",i);
\end{verbatim*}
\end{example}

\cmdindex{verb}
要排版简短的代码或关键字,可使用 \cmd{verb} 命令:
\begin{command}
\cmd{verb}\Arg{delim}\Arg{code}\Arg{delim}
\end{command}

\Arg{delim} 标明代码的分界位置,前后必须一致,除字母、空格或星号外,可任意选择使得不与代码本身冲突,习惯上使用 \texttt| 符号。

同 \env{verbatim} 环境,\cmd{verb} 后也可以带一个星号,以显示空格:
\begin{example}
\verb|\LaTeX| \\
\verb+(a || b)+ \verb*+(a || b)+
\end{example}

\cmd{verb} 命令对符号的处理比较复杂,一般\textbf{不能用在其它命令的参数里},否则多半会出错。

\pkg{verbatim} 宏包优化了 \env{verbatim} 环境的内部命令,并提供了 \cmd{verbatiminput} 命令用来直接读入文件生成代码环境。
\pkg{fancyvrb} 宏包提供了可定制格式的 \env{Verbatim} 环境;\pkg{listings} 宏包更进一步,可生成关键字高亮的代码环境,
支持各种程序设计语言的语法和关键字。详情请参考各自的帮助文档。

\section{表格}\label{sec:tabular}

\pinyinindex{biaoge}{表格|(}
\LaTeX{} 里排版表格不如 Word 等所见即所得的工具简便和自由,不过对于不太复杂的表格来讲,完全能够胜任。

\envindex{tabular}
\cmdindex{hline}
\index{&@\texttt\& (\textit{单元格/对齐})}
\index{\@\crcmd{} (\textit{换行})}
排版表格最基本的 \env{tabular} 环境用法为:
\begin{command}
\cmd{begin}\marg*{tabular}\oarg{align}\marg{column-spec}\\
 \Arg{item1} \texttt\& \Arg{item2} \texttt\& \ldots\ \crcmd \\
 \cmd{hline} \\
 \Arg{item1} \texttt\& \Arg{item2} \texttt\& \ldots\ \crcmd \\
\cmd{end}\marg*{tabular}
\end{command}
其中 \Arg{column-spec} 是列格式标记,在接下来的内容将仔细介绍;\texttt\& 用来分隔单元格;
\crcmd{} 用来换行;\cmd{hline} 用来在行与行之间绘制横线。

直接使用 \env{tabular} 环境的话,会\textbf{和周围的文字混排}。\env{tabular} 环境可带一个可选参数 \Arg{align} 控制垂直对齐(默认是垂直居中):
\begin{example}
\begin{tabular}{|c|}
  center-\\ aligned \\
\end{tabular},
\begin{tabular}[t]{|c|}
  top-\\ aligned \\
\end{tabular},
\begin{tabular}[b]{|c|}
  bottom-\\ aligned\\
\end{tabular} tabulars.
\end{example}

但是通常情况下 \env{tabular} 环境很少与文字直接混排,而是会放在 \env{table} 浮动体环境中,并用 \cmd{caption} 命令加标题。

\subsection{列格式}\label{subsec:tabular-cols}

\env{tabular} 环境使用 \Arg{column-spec} 参数指定表格的列数以及每列的格式。基本的列格式见表 \ref{tbl:table-column-spec}。
\begin{table}[htp]
\centering
\caption{\LaTeX{} 表格列格式。}\label{tbl:table-column-spec}
\begin{tabular}{*{2}{l}}
 \hline
 \textbf{列格式} & \textbf{说明} \\
 \hline
 \ttfamily l/c/r          & 单元格内容左对齐/居中/右对齐,不折行 \\
 \ttfamily p\marg{width}  & 单元格宽度固定为 \Arg{width},可自动折行 \\
 \ttfamily |              & 绘制竖线 \\
 \ttfamily @\marg{string} & 自定义内容 \Arg{string} \\
 \hline
\end{tabular}
\end{table}

\begin{example}
\begin{tabular}{lcr|p{6em}}
  \hline
  left & center & right
       & par box with fixed width\\
  L    & C      & R     & P \\
 \hline
\end{tabular}
\end{example}

表格中每行的单元格数目不能多于列格式里 \texttt{l/c/r/p} 的总数(可以少于这个总数),否则出错。

\texttt{@} 格式可在单元格前后插入任意的文本,但同时它也消除了单元格前后额外添加的间距。
\texttt{@} 格式可以适当使用以充当“竖线”。特别地,\texttt{@}\marg*{} 可直接用来消除单元格前后的间距:
\begin{example}
\begin{tabular}{@{} r@{:}lr @{}}
  \hline
  1  & 1 & one \\
  11 & 3 & eleven \\
  \hline
\end{tabular}
\end{example}

另外 \LaTeX{} 还提供了简便的将格式参数重复的写法 \texttt*\marg{n}\marg{column-spec},比如以下两种写法是等效的:
\begin{verbatim}
\begin{tabular}{|c|c|c|c|c|p{4em}|p{4em}|}
\begin{tabular}{|*{5}{c|}*{2}{p{4em}|}}
\end{verbatim}

\pkgindex{array}
有时需要为整列修饰格式,比如整列改变为粗体,如果每个单元格都加上 \cmd{bfseries} 命令会比较麻烦。
\pkg{array} 宏包提供了辅助格式 \texttt> 和 \texttt<,用于给列格式前后加上修饰命令:
\begin{example}
\begin{tabular}{>{\itshape}r<{*}l}
  \hline
  italic & normal \\
  column & column \\
  \hline
\end{tabular}
\end{example}

辅助格式甚至支持插入 \cmd{centering} 等命令改变 \texttt{p} 列格式的对齐方式,
一般还要加额外的命令 \cmd{array\-back\-slash} 以免出错%
\footnote{\cmd{centering} 等对齐命令会破坏表格环境里 \crcmd{} 换行命令的定义,
\cmd{array\-back\-slash} 用来恢复之。如果不加 \cmd{array\-back\-slash} 命令,
也可以用 \cmd{tabular\-newline} 命令代替原来的 \crcmd{} 实现表格换行。}:
\begin{example}
\begin{tabular}
{>{\centering\arraybackslash}p{9em}}
  \hline
  Some center-aligned long text. \\
  \hline
\end{tabular}
\end{example}

\pkg{array} 宏包还提供了类似 \texttt{p} 格式的 \texttt{m} 格式和 \texttt{b} 格式,
三者分别在垂直方向上靠顶端对齐、居中以及底端对齐。
\begin{example}
\newcommand\txt
  {a b c d e f g h i}
\begin{tabular}{cp{2em}m{2em}b{2em}}
  \hline
  pos & \txt & \txt & \txt \\
  \hline
\end{tabular}
\end{example}

\subsection{列宽}\label{subsec:colwidth}

在控制列宽方面,\LaTeX{} 表格有着明显的不足:\texttt{l/c/r} 格式的列宽是由文字内容的自然宽度决定的,
而 \texttt{p} 格式给定了列宽却不好控制对齐(可用 \pkg{array} 宏包的辅助格式),
更何况列与列之间通常还有间距,所以直接生成给定总宽度的表格并不容易。

\envindex{tabular*}
\LaTeX{} 本身提供了 \env{tabular*} 环境用来排版定宽表格,但是不太方便使用,
比如要用到 \texttt{@} 格式插入额外命令,令单元格之间的间距为 \cmd{fill},但即使这样仍然有瑕疵:
\begin{example}
\begin{tabular*}{14em}%
{@{\extracolsep{\fill}}|c|c|c|c|}
  \hline
  A & B & C & D \\ \hline
  a & b & c & d \\ \hline
\end{tabular*}
\end{example}

\pkgindex{tabularx}
\envindex[tabularx]{tabularx}
\pkg{tabularx} 宏包为我们提供了方便的解决方案。它引入了一个 \texttt{X} 列格式,类似 \texttt{p} 列格式,
不过会根据表格宽度自动计算列宽,多个 \texttt{X} 列格式平均分配列宽。
\texttt{X} 列格式也可以用 \pkg{array} 里的辅助格式修饰对齐方式:
\begin{example}
\begin{tabularx}{14em}%
{|*{4}{>{\centering\arraybackslash}X|}}
  \hline
  A & B & C & D \\ \hline
  a & b & c & d \\ \hline
\end{tabularx}
\end{example}

\subsection{横线}\label{subsec:hline}

\cmdindex{hline,cline}
我们已经在之前的例子见过许多次绘制表格线的 \cmd{hline} 命令。另外 \cmd{cline}\marg*{\Arg{i}-\Arg{j}} 用来绘制跨越部分单元格的横线:
\begin{example}
\begin{tabular}{|c|c|c|}
  \hline
  4 & 9 & 2 \\ \cline{2-3}
  3 & 5 & 7 \\ \cline{1-1}
  8 & 1 & 6 \\ \hline
\end{tabular}
\end{example}

\pkgindex{booktabs}
\cmdindex[booktabs]{toprule,midrule,bottomrule,cmidrule}
在科技论文排版中广泛应用的表格形式是三线表,形式干净简明。
三线表由 \pkg{booktabs} 宏包支持,它提供了 \cmd{toprule}、\cmd{midrule} 和 \cmd{bottomrule} 命令用以排版三线表的三条线,
以及和 \cmd{cline} 对应的 \cmd{cmidrule}。除此之外,最好不要用其它横线以及竖线:
\begin{example}
\begin{tabular}{cccc}
  \toprule
   & \multicolumn{3}{c}{Numbers} \\
  \cmidrule{2-4}
           & 1 & 2 & 3 \\
  \midrule
  Alphabet & A & B & C \\
  Roman    & I & II& III \\
  \bottomrule
\end{tabular}
\end{example}

\subsection{合并单元格}\label{subsec:tabular-multicol}

\LaTeX{} 是一行一行排版表格的,横向合并单元格较为容易,由 \cmd{multi\-column} 命令实现:
\begin{command}
\cmd{multicolumn}\marg{n}\marg{column-spec}\marg{item}
\end{command}
其中 \Arg{n} 为要合并的列数,\Arg{column-spec} 为合并单元格后的列格式,只允许出现一个 \texttt{l/c/r} 或 \texttt{p} 格式。
如果合并前的单元格前后带表格线 \texttt|,合并后的列格式也要带 \texttt| 以使得表格的竖线一致。
\begin{example}
\begin{tabular}{|c|c|c|}
  \hline
  1 & 2 & Center \\ \hline
  \multicolumn{2}{|c|}{3} &
  \multicolumn{1}{r|}{Right} \\ \hline
  4 & \multicolumn{2}{c|}{C} \\ \hline
\end{tabular}
\end{example}

上面的例子还体现了,形如 \cmd{multicolumn}\marg*{1}\marg{column-spec}\marg{item} 的命令\textbf{可以用来修改某一个单元格的列格式。}

\pkgindex{multirow}
\cmdindex[multirow]{multirow}
纵向合并单元格需要用到 \pkg{multirow} 宏包提供的 \cmd{multirow} 命令:
\begin{command}
\cmd{multirow}\marg{n}\marg{width}\marg{item}
\end{command}
\Arg{width} 为合并后单元格的宽度,可以填 \texttt{*} 以使用自然宽度。

我们看一个结合 \cmd{cline}、\cmd{multi\-column} 和 \cmd{multi\-row} 命令的例子:
\begin{example}
\begin{tabular}{ccc}
  \hline
  \multirow{2}{*}{Item} &
    \multicolumn{2}{c}{Value} \\
  \cline{2-3}
    & First & Second \\ \hline
  A & 1     & 2 \\ \hline
\end{tabular}
\end{example}

\subsection{嵌套表格}\label{subsec:tabular-embed}

相对于合并单元格,拆分单元格对于 \LaTeX{} 来说并非易事。在单元格中嵌套一个小表格可以起到“拆分单元格”的效果。
在以下的例子中,注意要用 \cmd{multi\-column} 命令配合 \texttt{@\{\}} 格式把单元格的额外边距去掉,使得嵌套的表格线能和外层的表格线正确相连:

\begin{example}
\begin{tabular}{|c|c|c|}
 \hline
 a & b & c \\ \hline
 a & \multicolumn{1}{@{}c@{}|}
 {\begin{tabular}{c|c}
   e & f \\ \hline
   e & f \\
  \end{tabular}}
       & c \\ \hline
 a & b & c \\ \hline
\end{tabular}
\end{example}

\pkgindex{makecell}
如果不需要为“拆分的单元格”画线,并且只在垂直方向“拆分”的话,\pkg{makecell} 宏包提供的 \cmd{make\-cell} 命令是一个简单的解决方案:

\begin{example}
\begin{tabular}{|c|c|}
 \hline
 a & \makecell{d1 \\ d2} \\
 \hline
 b & c \\
 \hline
\end{tabular}
\end{example}

\subsection{行距控制}\label{subsec:tabular-colht}

\cmdindex{arraystretch}
\LaTeX{} 生成的表格看起来通常比较紧凑。修改参数 \cmd{array\-stretch} 可以得到行距更加宽松的表格
(相关命令参考 \ref{subsec:newcmd} 小节):
\begin{example}
\renewcommand\arraystretch{1.8}
\begin{tabular}{|c|}
  \hline
  Really loose \\ \hline
  tabular rows.\\ \hline
\end{tabular}
\end{example}

\index{\@\crcmd{} (\textit{换行})}
另一种增加间距的办法是给换行命令 \crcmd{} 添加可选参数,在这一行下面加额外的间距,适合用于在行间不加横线的表格:
\begin{example}
\begin{tabular}{c}
  \hline
  Head lines \\[6pt]
  tabular lines \\
  tabular lines \\ \hline
\end{tabular}
\end{example}

但是这种换行方式的存在导致了一个缺陷——\textbf{表格的首个单元格不能直接使用中括号 \texttt{[]}},
否则 \crcmd{} 往往会将下一行的中括号当作自己的可选参数,因而出错。如果要使用中括号,应当放在花括号 \marg*{} 里面。
或者也可以选择将换行命令写成 \crcmd\texttt{[0pt]}。

\pinyinindex{biaoge}{表格|)}

\section{图片}\label{sec:figures}

\pkgindex{graphicx}

\LaTeX{} 本身不支持插图功能,需要由 \pkg{graphicx} 宏包辅助支持。

使用 \texttt{latex + dvipdfmx} 编译命令时,调用 \pkg{graphicx} 宏包时要指定 \texttt{dvipdfmx} 选项%
\footnote{早期常使用 \texttt{latex + dvips} 组合命令,后者将 \texttt{.dvi} 文件转为 \texttt{.ps} 文件(PostScript),
可进一步通过 \texttt{ps2pdf} 工具生成 PDF。\texttt{dvips} 和 \texttt{dvipdfmx} 在图形、颜色、超链接等功能的实现上有差别,而 \LaTeX{} 无法识别
用户是用 \texttt{dvips} 还是 \texttt{dvipdfmx},所以要指定选项(缺省为 \texttt{dvips})。
\ref{sec:hyperlinks} 节中的 \pkg{hyperref} 宏包同理。};而使用 \texttt{pdflatex} 或 \texttt{xelatex} 命令编译时不需要。

读者可能听说过“\LaTeX{} 只能插入 \texttt{.eps} 格式的图片,需要把 \texttt{.jpg} 转成 \texttt{.eps} 格式”的观点。
\LaTeX{} 发展到今天,这个观点早已过时。事实上不同的编译命令支持的图片格式范围各异,见表 \ref{tbl:figure-format}。
这个表格也能解答诸如“为什么 \texttt{.eps} 格式图片在 \texttt{pdflatex} 编译命令下出错”之类的问题。本表格也再一次说明,使用
\texttt{xelatex} 命令是笔者最推荐的方式。

\begin{table}[htp]
\centering
\caption{各种编译方式支持的主流图片格式。}\label{tbl:figure-format}
\begin{tabular}{*{3}{l}}
 \hline
 \textbf{格式}  & \textbf{矢量图} & \textbf{位图} \\
 \hline
 \texttt{latex + dvipdfmx}           & \texttt{.eps}      & n/a \\
 \quad $\llcorner$(调用 \pkg{bmpsize} 宏包 )   & \texttt{.eps .pdf}     & \texttt{.jpg .png .bmp} \\[.3\baselineskip]
 \texttt{pdflatex}                   & \texttt{.pdf}      & \texttt{.jpg .png} \\
 \quad $\llcorner$(调用 \pkg{epstopdf} 宏包)   & \texttt{.pdf .eps} & \texttt{.jpg .png} \\[.3\baselineskip]
 \texttt{xelatex}                    & \texttt{.pdf .eps} & \texttt{.jpg .png .bmp} \\
 \hline
\end{tabular}
\begin{quote}\footnotesize
注:在较新的 \TeX{} 发行版中,\texttt{latex + dvipdfmx} 和 \texttt{pdf\-latex} 命令可不依赖宏包,支持原来需要宏包扩展的图片格式
(但 \texttt{pdf\-latex} 命令仍不支持 \texttt{.bmp} 格式的位图)。
\end{quote}
\end{table}

\cmdindex[graphicx]{includegraphics}
在调用了 \pkg{graphicx} 宏包以后,就可以使用 \cmd{include\-graphics} 命令加载图片了:
\begin{command}
\cmd{includegraphics}\oarg{options}\marg{filename}
\end{command}

\cmdindex[graphicx]{graphicspath}
其中 \Arg{filename} 为图片文件名,与 \cmd{include} 命令的用法类似,文件名可能需要用相对路径或绝对路径表示(见 \ref{sec:latex-multi-files} 节)。
图片文件的扩展名一般可不写。另外一定要注意,\textbf{文件名里既不要有空格(类似 \cmd{include}),也不要有多余的英文点号},否则宏包在解析文件名的过程中会出错。

另外 \pkg{graphicx} 宏包还提供了 \cmd{graphics\-path} 命令,用于声明一个或多个图片文件存放的目录,
使用这些目录里的图片时可不用写路径:
\begin{verbatim}
% 假设主要的图片放在 figures 子目录下,标志放在 logo 子目录下
\graphicspath{{figures/}{logo/}}
\end{verbatim}

\cmd{includegraphics} 命令的可选参数 \Arg{options} 支持 \Arg{key}=\Arg{value} 形式赋值,常用的参数如下:
\begin{table}[htp]
\centering
\caption{\cmd{includegraphics} 命令的可选参数。}\label{tbl:graphics-options}
\begin{tabular}{lp{18em}}
 \hline
 \textbf{参数} & \textbf{含义} \\
 \hline
 \texttt{width=}\Arg{width}    &  将图片缩放到宽度为 \Arg{width} \\
 \texttt{height=}\Arg{height}  &  将图片缩放到高度为 \Arg{height} \\
 \texttt{scale=}\Arg{scale}    &  将图片相对于原尺寸缩放 \Arg{scale} 倍 \\
 \texttt{angle=}\Arg{angle}    &  令图片逆时针旋转 \Arg{angle} 度 \\
 \hline
\end{tabular}
\end{table}

\section{盒子}\label{sec:box}

盒子是 \LaTeX{} 排版的基础单元,虽然解释上去有些抽象:每一行是一个盒子,
里面的文字从左到右依次排列;每一页也是一个盒子,各行文字从上到下依次排布……颇有一些活字印刷术的味道。

不管如何,\LaTeX{} 提供了一些命令让我们生成一些有特定用途的盒子。

\subsection{水平盒子}\label{subsec:mbox}

\cmdindex{mbox,makebox}
生成水平盒子的命令如下:
\begin{command}
\cmd{mbox}\marg*{\ldots} \\
\cmd{makebox}\oarg{width}\oarg{align}\marg*{\ldots}
\end{command}

\cmd{mbox} 生成一个基本的水平盒子,内容只有一行,不允许分段(除非嵌套其它盒子,比如后文的垂直盒子)。
外表看上去,\cmd{mbox} 的内容与正常的文本无二,不过断行时文字不会从盒子里断开。

\cmd{makebox} 更进一步,可以加上可选参数用于控制盒子的宽度 \Arg{width},以及内容的对齐方式 \Arg{align},
可选居中 \texttt{c}(默认值)、左对齐 \texttt{l}、右对齐 \texttt{r} 和分散对齐 \texttt{s}%
\footnote{分散对齐方式强行拉开单词的间距,往往会报 \texttt{Underfull} \cmd{hbox} 的消息。}。
\begin{example}
|\mbox{Test some words.}|\\
|\makebox[10em]{Test some words.}|\\
|\makebox[10em][l]{Test some words.}|\\
|\makebox[10em][r]{Test some words.}|\\
|\makebox[10em][s]{Test some words.}|
\end{example}

\subsection{带框的水平盒子}\label{subsec:fbox}

\cmdindex{fbox,framebox}
\cmd{fbox} 和 \cmd{framebox} 让我们可以为水平盒子添加边框。使用的语法与 \cmd{mbox} 和 \cmd{makebox} 一模一样:
\begin{command}
\cmd{fbox}\marg*{\ldots} \\
\cmd{framebox}\oarg{width}\oarg{align}\marg*{\ldots}
\end{command}
\begin{example}
\fbox{Test some words.}\\
\framebox[10em][r]{Test some words.}
\end{example}

\cmdindex{setlength}
\cmdindex{fboxrule,fboxsep}
可以通过 \cmd{setlength} 命令(见 \ref{subsec:lengths} 小节)调节边框的宽度 \cmd{fbox\-rule} 和内边距 \cmd{fbox\-sep}:
\begin{example}
\framebox[10em][r]{Test box}\\[1ex]
\setlength{\fboxrule}{1.6pt}
\setlength{\fboxsep}{1em}
\framebox[10em][r]{Test box}
\end{example}

\subsection{垂直盒子}\label{subsec:parbox}

\cmdindex{parbox}
\envindex{minipage}
如果需要排版一个文字可以换行的盒子,\LaTeX{} 提供了两种方式:
\begin{command}
\cmd{parbox}\oarg{align}\oarg{height}\oarg{inner-align}\marg{width}\marg*{\ldots} \\[0.5ex]
\cmd{begin}\marg*{minipage}\oarg{align}\oarg{height}\oarg{inner-align}\marg{width} \\
\ldots \\
\cmd{end}\marg*{minipage}
\end{command}

其中 \Arg{align} 为盒子和周围文字的对齐情况(类似 \env{tabular} 环境);
\Arg{height} 和 \Arg{inner-align} 设置盒子的高度和内容的对齐方式,类似水平盒子 \cmd{makebox} 的设置,
不过 \Arg{inner-align} 接受的参数是顶部 \texttt{t}、底部 \texttt{b}、居中 \texttt{c} 和分散对齐 \texttt{s}。

\begin{example}
三字经:\parbox[t]{3em}%
{人之初 性本善 性相近 习相远}
\quad
千字文:
\begin{minipage}[b][8ex][t]{4em}
天地玄黄 宇宙洪荒
\end{minipage}
\end{example}

\cmdindex{footnote,footnotetext,footnotemark}
如果在 \env{minipage} 里使用 \cmd{footnote} 命令,生成的脚注会出现在盒子底部,编号是独立的,并且使用小写字母编号。
这也是 \env{minipage} 环境之被称为“迷你页”(Mini-page)的原因。
而在 \cmd{parbox} 里无法正常使用 \cmd{footnote} 命令,只能在盒子里使用 \cmd{foot\-note\-mark},在盒子外使用 \cmd{foot\-note\-text}。
\begin{example}
\fbox{\begin{minipage}{15em}%
  这是一个垂直盒子的测试。
  \footnote{脚注来自 minipage。}
\end{minipage}}
\end{example}

\subsection{标尺盒子}\label{subsec:rules}

\cmdindex{rule}
\cmd{rule} 命令用来画一个实心的矩形盒子,也可适当调整以用来画线(标尺):
\begin{command}
\cmd{rule}\oarg{raise}\marg{width}\marg{height}
\end{command}
\begin{example}
Black \rule{12pt}{4pt} box.

Upper \rule[4pt]{6pt}{8pt} and
lower \rule[-4pt]{6pt}{8pt} box.

A \rule[-.4pt]{3em}{.4pt} line.
\end{example}

\section{浮动体}\label{sec:float}

内容丰富的文章或者书籍往往包含许多图片和表格等内容。这些内容的尺寸往往太大,导致分页困难。
\LaTeX{} 为此引入了浮动体的机制,令大块的内容可以脱离上下文,放置在合适的位置。

\envindex{figure,table}
\LaTeX{} 预定义了两类浮动体环境 \env{figure} 和 \env{table}。习惯上 \env{figure} 里放图片,\env{table} 里放表格,
但并没有严格限制,可以在任何一个浮动体里放置文字、公式、表格、图片等等任意内容。

以 \env{table} 环境的用法举例,\env{figure} 同理:
\begin{command}
\cmd{begin}\marg*{table}\oarg{placement} \\
\ldots \\
\cmd{end}\marg*{table}
\end{command}

\Arg{placement} 参数提供了一些符号用来表示浮动体允许排版的位置,如 \texttt{hbp} 允许浮动体排版在当前位置、底部或者单独成页。
\env{table} 和 \env{figure} 浮动体的默认设置为 \texttt{tbp}。
\begin{table}[htp]
\centering
\caption{浮动体的位置参数。}\label{tbl:float-placement}
\begin{tabular}{*{2}{l}}
 \hline
 \textbf{参数} & \textbf{含义} \\
 \hline
 \texttt{h} & 当前位置(代码所处的上下文) \\
 \texttt{t} & 顶部 \\
 \texttt{b} & 底部 \\
 \texttt{p} & 单独成页 \\
 \texttt{!} & 在决定位置时忽视限制 \\
 \hline
\end{tabular}
\begin{quote}\footnotesize
注 1:排版位置的选取与参数里符号的顺序无关,\LaTeX{} 总是以 \texttt{h-t-b-p} 的优先级顺序决定浮动体位置。
也就是说 \texttt{[!htp]} 和 \texttt{[ph!t]} 没有区别。\par
注 2:限制包括浮动体个数(除单独成页外,默认每页不超过 3 个浮动体,其中顶部不超过 2 个,底部不超过 1 个)
以及浮动体空间占页面的百分比(默认顶部不超过 70\%,底部不超过 30\%)。
\end{quote}
\end{table}

\envindex{table*,figure*}
双栏排版环境下,\LaTeX{} 提供了 \env{table*} 和 \env{figure*} 环境用来排版跨栏的浮动体。
它们的用法与 \env{table} 和 \env{figure} 一样,不同之处为双栏的 \Arg{placement} 参数只能用 \texttt{tp} 两个位置。

\cmdindex{clearpage}
浮动体的位置选取受到先后顺序的限制。如果某个浮动体由于参数限制、空间限制等原因在当前页无法放置,就要推迟到之后处理,
并使得之后的同类浮动体一并推迟。\cmd{clearpage} 命令会在另起一页之前,先将所有推迟处理的浮动体排版成页,
此时 \texttt{htbp} 等位置限制被完全忽略。

\pkgindex{float}

\pkg{float} 宏包为浮动体提供了 \texttt{H} 位置参数,不与 \texttt{htbp} 及 \texttt{!} 混用。使用 \texttt{H} 位置参数时,
会取消浮动机制,将浮动体视为一般的盒子插入当前位置。这在一些特殊情况下很有用(如使用 \pkg{multicol} 宏包排版分栏内容的时候),
但尺寸过大的浮动体可能使得分页比较困难。

\subsection{浮动体的标题}\label{subsec:caption}

\cmdindex{caption,label}
图表等浮动体提供了 \cmd{caption} 命令加标题,并且自动给浮动体编号:
\begin{command}
\cmd{caption}\marg*{\ldots}
\end{command}
\cmd{caption} 的用法非常类似于 \cmd{section} 等命令,可以用带星号的命令 \cmd{caption*} 生成不带编号的标题,
也可以使用带可选参数的形式 \cmd{caption}\oarg*{\ldots}\marg*{\ldots},使得在目录里使用短标题。
\cmd{caption} 命令之后还可以紧跟 \cmd{label} 命令标记交叉引用。

\pkgindex{caption}
\cmd{caption} 生成的标题形如 ``Figure 1: \ldots''{}(\env{figure} 环境)或 ``Table 1: \ldots''{}(\env{table} 环境)。
可通过修改 \cmd{figurename} 和 \cmd{tablename} 的内容来修改标题的前缀(详见第 \ref{sec:latex-settings} 节)。
标题样式的定制功能由 \pkg{caption} 宏包提供,详见该宏包的帮助文档,在此不作赘述。

\cmdindex{listoftables,listoffigures}
\env{table} 和 \env{figure} 两种浮动体分别有各自的生成目录的命令:
\begin{command}
\cmd{listoftables} \\
\cmd{listoffigures}
\end{command}

它们类似 \cmd{tableofcontents} 生成单独的章节。

\subsection{并排和子图表}\label{subsec:subfig}

我们时常有在一个浮动体里面放置多张图的用法。最简单的用法就是直接并排放置,
也可以通过分段或者换行命令 \crcmd{} 排版多行多列的图片。以下为示意代码,效果大致如图 \ref{fig:parallel-fig} 所示。
\begin{verbatim}
\begin{figure}[htbp]
  \centering
  \includegraphics[width=...]{...}
  \qquad
  \includegraphics[width=...]{...} \\[..pt]
  \includegraphics[width=...]{...}
  \caption{...}
\end{figure}
\end{verbatim}

\begin{figure}[htp]
  \centering
  \fcolorbox[gray]{0}{0.96}{\parbox{10em}{\vrule width 0pt height 10ex\hfil}}
  \qquad
  \fcolorbox[gray]{0}{0.96}{\parbox{10em}{\vrule width 0pt height 12ex\hfil}}
  \par\bigskip
  \fcolorbox[gray]{0}{0.96}{\parbox{20em}{\vrule width 0pt height 12ex\hfil}}
  \caption{并排放置图片的示意。}\label{fig:parallel-fig}
\end{figure}

由于标题是横跨一行的,用 \cmd{caption} 命令为每个图片单独生成标题就需要借助前文提到的 \cmd{parbox}
或者 \env{minipage} 环境,将标题限制在盒子内。效果见图 \ref{fig:parallel-cap1} 和图 \ref{fig:parallel-cap2}。
\begin{verbatim}
\begin{figure}[htbp]
  \centering
  \begin{minipage}{...}
    \centering
    \includegraphics[width=...]{...}
    \caption{...}
  \end{minipage}
  \qquad
  \begin{minipage}{...}
    \centering
    \includegraphics[width=...]{...}
    \caption{...}
  \end{minipage}
\end{figure}
\end{verbatim}

\begin{figure}[htp]
  \centering
  \begin{minipage}{12em}
    \centering
    \fcolorbox[gray]{0}{0.96}{\parbox{10em}{\vrule width 0pt height 12ex\hfil}}
    \caption{并排图1。}\label{fig:parallel-cap1}
  \end{minipage}
  \qquad
  \begin{minipage}{12em}
    \centering
    \fcolorbox[gray]{0}{0.96}{\parbox{10em}{\vrule width 0pt height 12ex\hfil}}
    \caption{并排图2。}\label{fig:parallel-cap2}
  \end{minipage}
\end{figure}

\pkgindex{subfig}
\cmdindex[subfig]{subfloat}
当我们需要更进一步,给每个图片定义小标题时,就要用到 \pkg{subfig} 宏包的功能了。
这里仅举一例,效果见图 \ref{fig:subfig}。更详细的用法请参考 \pkg{subfig} 宏包的帮助文档。
\begin{verbatim}
\begin{figure}[htbp]
  \centering
  \subfloat[...]{\label{sub-fig-1}% 为子图加交叉引用
   \begin{minipage}{...}
    \centering
    \includegraphics[width=...]{...}
   \end{minipage}
  }
  \qquad
  \subfloat[...]{%
  \begin{minipage}{...}
    \centering
    \includegraphics[width=...]{...}
  \end{minipage}
  }
  \caption{...}
\end{figure}
\end{verbatim}

\begin{figure}[htp]
  \centering
  \subfloat[并排子图1]{%
  \begin{minipage}{12em}
    \centering
    \fcolorbox[gray]{0}{0.96}{\parbox{10em}{\vrule width 0pt height 12ex\hfil}}
  \end{minipage}}
  \qquad
  \subfloat[并排子图2]{%
  \begin{minipage}{12em}
    \centering
    \fcolorbox[gray]{0}{0.96}{\parbox{10em}{\vrule width 0pt height 12ex\hfil}}
  \end{minipage}}
  \caption{使用 \pkg{subfig} 宏包的 \cmd{subfloat} 命令排版子图。}\label{fig:subfig}
\end{figure}

\endinput
